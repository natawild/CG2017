\chapter{Conclusão}

Nesta fase procedeu-se a vários ajustes a nível da estrutura das classes, mas também otimizaçõess de código. Assumir isto como uma prioridade significou aliviar a implementaçãoo de novas funcionalidades futuras que podem
ser mais exigentes a nível de processamento. Culmatou-se desta forma alguns dos problemas de desorganizaçãoo do código apresentados anteriormente.

Resolveu-se o problema de recarregamento dos ficheiros sempre que se fazia render. Foram realizadas algumas alterações a nivel da câmara, dando um maior controle ao utilizador, permitindo, para além do que já podia realizar na primeira fase, fazer zoom in e zoom out e melhorou-se a rotação da câmara. 

Continua a não ser possivel diferenciar por cores as diferentes figuras, sendo estes um dos aspectos que queremos melhorar nas seguintes fases.
Por outro lado, tentamos estar atentos a possiveis  problemas de otimização, tal como a leitura repetida de ficheiros iguais. Por exemplo, para dois planetas definidos pelo mesmo ficheiro, a estrutura carregada terá essa informaçã repetida, e o ficheiro será lido duas vezes. 

Em suma, achamos que, com esta fase 2, definiu-se as estruturas base a utilizar e simplificou-se a introdução de novas funcionalidades tendo-se melhorado as funcionalidades já existentes.
