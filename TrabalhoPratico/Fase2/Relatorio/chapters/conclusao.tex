\chapter{Conclusão}

Nesta fase procederam-se a vários ajustes a nível da estrutura das classes, mas também otimizações de código. Assumir isto como uma prioridade significou aliviar a implementação de novas funcionalidades futuras que podem ser mais exigentes a nível de processamento. Culmatou-se desta forma alguns dos problemas de desorganização do código apresentados anteriormente.

Foram realizadas algumas alterações a nivel da câmara, dando um maior controle ao utilizador, permitindo, para além do que já podia realizar na primeira fase, fazer zoom in e zoom out e melhorou-se a rotação da câmara. 

Por outro lado, tentamos estar atentos a possíveis  problemas de otimização, tal como a leitura repetida de ficheiros iguais. Por exemplo, para dois planetas definidos pelo mesmo ficheiro, a estrutura carregada poderia ter essa informação repetida, e o ficheiro poderia ser lido duas vezes, o que não está a acontecer. 

Em suma, achamos que, com esta fase 2, definiu-se as estruturas base a utilizar e simplificou-se a introdução de novas funcionalidades tendo-se melhorado as funcionalidades já existentes.

