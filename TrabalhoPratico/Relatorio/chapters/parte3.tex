\chapter{Gerador}
\label{cap:p3}

\section{Objetivos}

O gerador é uma aplicação que é capaz de receber e interpretar pedidos do utilizador para desenho de figuras e gerar um ficheiro .3d com os pontos correspondentes à figura pedida.
Nesta primeira etapa, o objetivo implementar o suporte à criação de um plano, uma caixa, um cone e uma esfera.

\section{Programa principal}

O pseudo-código do programa principal é o seguinte:


\begin{Verbatim}
int main(int argc, char** argv) {
	Declara figura onde vao ser guardados os pontos
	figuraCriada = false;
	
	if (figura pedida == plano) {
		Lê e interpreta parametros
		Chama funçao da figura para desenhar os pontos 
		de um plano
		Grava pontos em ficheiro
		figuraCriada = true;
	}
	
	if (figura pedida == caixa) {
		Lê e interpreta parametros
		Chama funçao da figura para desenhar os pontos
		de uma caixa
		Grava pontos em ficheiro
		figuraCriada = true;
	}
		
	if (figura pedida == cone) {
		Lê e interpreta parametros
		Chama funçao da figura para desenhar os pontos 
		de um cone
		Grava pontos em ficheiro
		figuraCriada = true;
	}

	if (figura pedida == esfera) {
		Lê e interpreta parametros
		Chama funçao da figura para desenhar os pontos
		 de uma esfera
		Grava pontos em ficheiro
		figuraCriada = true;
	}
	
	
	if (!figuraCriada) {
		if (o programa for corrido sem argumentos) {
			Informar utilizador que programa 
			foi corrido sem argumentos
		}
		else {
			Informar utilizador que nao foi possivel
			 criar a figura
		}
		Mostra mensagem de ajuda com sintaxe dos comandos
	}
	
	
	return 0;
}

\end{Verbatim}

\newpage
\section{Primitivas}

Nesta secção apresentam-se os comandos correspondentes aos pedidos de figuras que o utilizador pode fazer e é explicada a sua implementação.

\subsection{Gerar Planos}
\label{p3:planos}

Para gerar um plano, o utilizador deve efetuar o comando com a seguinte sintaxe:

\begin{Verbatim}
Gerador plane comprimento largura ficheiro
\end{Verbatim}

O resultado deste comando é a criação de um plano em XZ centrado no ponto (0,0,0) com o comprimento e largura indicados.

\subsubsection{Plano em XZ (Y=constante)}

Um plano é formado por 2 triângulos. Com a informação de 4 pontos é possível desenhar esses triângulos. A figura \ref{p1:fig:p3_planoY} representa um plano em XZ. De notar que é possível considerar dois triângulos: o triângulo formado por [ABC] e o triângulo formado por [CDA]

\begin{figure}[<+htpb+>]
	\centering
	\includegraphics[scale=0.5]{imagens/p3_planoY.png}
	\caption{Plano em XZ centrado na origem}
	\label{p1:fig:p3_planoY}
\end{figure}

Estando o plano centrado na origem, e sabendo o comprimento e largura do plano, conclui-se que as coordenadas dos pontos da figura \ref{p1:fig:p3_planoY} são as seguintes:

\begin{Verbatim}
A(-comprimento/2, 0, -largura/2)
B(-comprimento/2, 0, largura/2)
C(comprimento/2, 0, largura/2)
P(0, 0, 0)
\end{Verbatim}

É também fácil exprimir as coordenadas de um plano não necessariamente centrado na origem, em função do seu centro P:

\begin{Verbatim}
A(-comprimento/2 + px, 0 + py, -largura/2 + pz)
B(-comprimento/2 + px, 0 + py, largura/2 + pz)
C(comprimento/2 + px, 0 + py, largura/2 + pz)
P(px, py, pz)
\end{Verbatim}

A ordem pela qual se adiciona os pontos à figura determina para que lado ela fica virada. Se quisermos que o plano fique virado para o sentido positivo do eixo dos Y, deve-se colocar os pontos pela seguinte ordem: A-B-C (1º triângulo), seguido de C-D-A (2º triângulo). Se quisermos que o plano fique virado para o sentido negativo do eixo dos Y, deve-se colocar os pontos pela seguinte ordem: A-D-C (1º triângulo), seguido de C-B-A (2º triângulo).

A função responsável por implementar este algoritmo é a função \textit{criaPlanoEmY()}, cujo pseudo-código se apresenta de seguida:

\begin{Verbatim}
Figura& criaPlanoEmY(Ponto3D centroPlano, float comprimento, 
float largura, int orientacao) {

	Calcula coordendas dos pontos A,B,C e D
	
	if (orientacao == 1) {
		Coloca pontos pela ordem A-B-C-C-D-A
	}
	else {
		Coloca pontos pela ordem A-D-C-C-B-A
	}
	
	return *this;
}
\end{Verbatim}

De notar que esta função além do comprimento e largura, recebe ainda o centro do plano e a orientação do mesmo.

\begin{figure}[<+htpb+>]
	\centering
	\includegraphics[scale=0.5]{imagens/p3_plano_4_2.png}
	\caption{Exemplo de plano em XZ gerado, com 4 de comprimento e 2 de largura}
	\label{p1:fig:p3_plano_4_2}
\end{figure}

\subsubsection{Plano em XY (Z=constante)}

Embora apenas fosse pedido que o gerador tivesse a capacidade de gerar um plano em XZ, considerou-se útil disponibilizar também uma primitiva para criar planos em XY.
Um plano é formado por 2 triângulos. Com a informação de 4 pontos é possível desenhar esses triângulos. A figura \ref{p1:fig:p3_planoZ} representa um plano em XY. De notar que é possível considerar dois triângulos: o triângulo formado por [ABC] e o triângulo formado por [CDA]

\begin{figure}[<+htpb+>]
	\centering
	\includegraphics[scale=0.5]{imagens/p3_planoZ.png}
	\caption{Plano em XZ centrado na origem}
	\label{p1:fig:p3_planoZ}
\end{figure}

Estando o plano centrado na origem, e sabendo o comprimento e largura do plano, conclui-se que as coordenadas dos pontos da figura \ref{p1:fig:p3_planoZ} são as seguintes:

\begin{Verbatim}
A(-comprimento/2 , altura/2, 0)
B(-comprimento/2 , -altura/2, 0)
C(comprimento/2 , -altura/2, 0)
D(comprimento/2 , altura/2, 0)
P(0,0,0)
\end{Verbatim}

É também fácil exprimir as coordenadas de um plano não necessariamente centrado na origem, em função do seu centro P:

\begin{Verbatim}
A(-comprimento/2 + px, altura/2 + py, 0 + pz)
B(-comprimento/2 + px, -altura/2 + py, 0 + pz)
C(comprimento/2 + px, -altura/2 + py, 0 + pz)
D(comprimento/2 +px, altura/2 + py, 0 + pz)
P(px, py, pz)
\end{Verbatim}

A ordem pela qual se adiciona os pontos à figura determina para que lado ela fica virada. Se quisermos que o plano fique virado para o sentido positivo do eixo dos Y, deve-se colocar os pontos pela seguinte ordem: A-B-C (1º triângulo), seguido de C-D-A (2º triângulo). Se quisermos que o plano fique virado para o sentido negativo do eixo dos Y, deve-se colocar os pontos pela seguinte ordem: A-D-C (1º triângulo), seguido de C-B-A (2º triângulo).

A função responsável por implementar este algoritmo é a função \textit{criaPlanoEmZ()}, cujo pseudo-código se apresenta de seguida:

\begin{Verbatim}
Figura& criaPlanoEmZ(Ponto3D centroPlano, float comprimento, 
float altura, int orientacao) {

Calcula coordendas dos pontos A,B,C e D

if (orientacao == 1) {
Coloca pontos pela ordem A-B-C-C-D-A
}
else {
Coloca pontos pela ordem A-D-C-C-B-A
}

return *this;
}
\end{Verbatim}

De notar que esta função além do comprimento e altura, recebe ainda o centro do plano e a orientação do mesmo.

\subsubsection{Plano em YZ (X=constante)}

Embora apenas fosse pedido que o gerador tivesse a capacidade de gerar um plano em XZ, considerou-se útil disponibilizar também uma primitiva para criar planos em YZ.
Um plano é formado por 2 triângulos. Com a informação de 4 pontos é possível desenhar esses triângulos. A figura \ref{p1:fig:p3_planoX} representa um plano em YZ. De notar que é possível considerar dois triângulos: o triângulo formado por [ABC] e o triângulo formado por [CDA]

\begin{figure}[<+htpb+>]
	\centering
	\includegraphics[scale=0.5]{imagens/p3_planoX.png}
	\caption{Plano em XZ centrado na origem}
	\label{p1:fig:p3_planoX}
\end{figure}

Estando o plano centrado na origem, e sabendo o comprimento e largura do plano, conclui-se que as coordenadas dos pontos da figura \ref{p1:fig:p3_planoX} são as seguintes:

\begin{Verbatim}
A(0, altura/2, comprimento/2)
B(0, -altura/2, comprimento/2)
C(0, -altura/2, -comprimento/2)
D(0, altura/2, -comprimento/2)
P(0,0,0)
\end{Verbatim}

É também fácil exprimir as coordenadas de um plano não necessariamente centrado na origem, em função do seu centro P:

\begin{Verbatim}
A(0 + px, altura/2 + py, comprimento/2 + pz)
B(0 + px, -altura/2 + py, comprimento/2 + pz)
C(0 + px, -altura/2 + py, -comprimento/2 + pz)
D(0 + px, altura/2 + py, -comprimento/2 + pz)
P(px, py, pz)
\end{Verbatim}

A ordem pela qual se adiciona os pontos à figura determina para que lado ela fica virada. Se quisermos que o plano fique virado para o sentido positivo do eixo dos Y, deve-se colocar os pontos pela seguinte ordem: A-B-C (1º triângulo), seguido de C-D-A (2º triângulo). Se quisermos que o plano fique virado para o sentido negativo do eixo dos Y, deve-se colocar os pontos pela seguinte ordem: A-D-C (1º triângulo), seguido de C-B-A (2º triângulo).

A função responsável por implementar este algoritmo é a função \textit{criaPlanoEmX()}, cujo pseudo-código se apresenta de seguida:

\begin{Verbatim}
Figura& criaPlanoEmX(Ponto3D centroPlano, float comprimento, 
float altura, int orientacao) {

Calcula coordendas dos pontos A,B,C e D

if (orientacao == 1) {
Coloca pontos pela ordem A-B-C-C-D-A
}
else {
Coloca pontos pela ordem A-D-C-C-B-A
}

return *this;
}
\end{Verbatim}

De notar que esta função além do comprimento e altura, recebe ainda o centro do plano e a orientação do mesmo.

\newpage
\subsection{Gerar Caixa}

Para gerar uma caixa, o utilizador deve efetuar o comando com a seguinte sintaxe:

\begin{Verbatim}
Gerador caixa comprimento largura altura ficheiro
\end{Verbatim}

O resultado deste comando é a criação de uma caixa centrada no ponto (0,0,0) com o comprimento, largura e altura indicados.

Na secção \ref{p3:planos} foram apresentadas as primitivas que permitiam fazer planos. Nomeadamente viu-se que essas primitivas permitiam geral planos em XY, YZ e XZ dado um centro e uma orientação.

Uma caixa é apenas um conjunto de planos. Assim, para gerar a caixa o que se fez foi gerar os seus planos usando as primitivas da secção \ref{p3:planos}. No entanto, para usar tais primitivas é necessário indicar um centro do plano e uma orientação para o mesmo. É por isso necessário calcular esses parâmetros antes de usar as primitivas dos planos.

Considere-se a caixa com as faces visíveis identificadas pelas letras A, B e C, conforme mostra a figura \ref{p1:fig:p3_Caixa}

\begin{figure}[<+htpb+>]
	\centering
	\includegraphics[scale=0.5]{imagens/p3_Caixa.png}
	\caption{Esquema representativo da caixa}
	\label{p1:fig:p3_Caixa}
\end{figure}

Considere-se ainda que a face oposta a A é designada por A', a face oposta a B por B' e a face oposta a C por C'.
Interessa agora saber as coordenadas dos centros de cada uma destas faces. Considerando a caixa centrada na origem, temos o seguinte:

\begin{Verbatim}
Centro A  = (0,altura/2,0)          Orientação = 1
Centro A' = (0,-altura/2,0)         Orientação = 0
Centro B  = (0,0,largura/2)         Orientação = 1
Centro B' = (0,0,-largura/2)        Orientação = 0
Centro C  = (comprimento/2,0,0)     Orientação = 1
Centro C' = (-comprimento/2,0,0)    Orientação = 0
\end{Verbatim}

O valor da orientação toma o valor 1 se o plano estiver virado no sentido positivo do eixo sobre o qual esta colocado.

É fácil verificar que caso a caixa não esteja centrada na origem mas esteja centrada num ponto P(px, py, pz) as coordenadas dos centros das faces serão:

\begin{Verbatim}
Centro A  = (0 + px,altura/2 + py,0 + pz)          Orientação = 1
Centro A' = (0 + px,-altura/2 + py,0 + pz)         Orientação = 0
Centro B  = (0 + px,0 + py,largura/2 + pz)         Orientação = 1
Centro B' = (0 + px,0 + py,-largura/2 + pz)        Orientação = 0
Centro C  = (comprimento/2 + px,0 + py,0 + pz)     Orientação = 1
Centro C' = (-comprimento/2 + px,0 + py,0 + pz)    Orientação = 0
\end{Verbatim}

A função da classe \textit{Figura} responsável por gerar os pontos de uma caixa é a função \textit{criaCaixa()} cujo pseudo-código se apresenta a seguir:

\begin{Verbatim}
Figura& criaCaixa(Ponto3D centroCaixa, 
	float dx, float dy, float dz) {
	
	Calcula coordendas centro A
	Cria plano com centro em A e orientacao = 1
	
	Calcula coordendas centro A'
	Cria plano com centro em A' e orientacao = 0
	
	Calcula coordendas centro B
	Cria plano com centro em B e orientacao = 1
	
	Calcula coordendas centro B'
	Cria plano com centro em B' e orientacao = 0
	
	Calcula coordendas centro C
	Cria plano com centro em C e orientacao = 1
	
	Calcula coordendas centro C'
	Cria plano com centro em C' e orientacao = 0
	
	return *this;
}

\end{Verbatim}

\begin{figure}[<+htpb+>]
	\centering
	\includegraphics[scale=0.5]{imagens/p3_caixa_6_3_4.png}
	\caption{Exemplo de caixa gerada, com 6 de comprimento, 3 de largura e 4 de altura}
	\label{p1:fig:p3_caixa_6_3_4}
\end{figure}

\newpage
\subsection{Gerar Círculo}
\label{p3:circulo}

Um círculo corresponde apenas a um conjunto de triângulos em que o centro do círculo é um ponto comum a todos os triângulos. Na figura \ref{p1:fig:p3_circulo} destacado a vermelho mostra-se um desses triângulos:

\begin{figure}[<+htpb+>]
	\centering
	\includegraphics[scale=0.5]{imagens/p3_circulo.png}
	\caption{Esquema representativo do círculo}
	\label{p1:fig:p3_circulo}
\end{figure}

Para se saber os pontos de um círculo foram usadas coordenadas polares. Para tal a classe \textit{CoordsPolares} foi usada (ver secção \ref{p1:coordsPolares}).

O número de triângulos de um círculo corresponde ao número de ``fatias'' que se pretende e com isso é possível calcular a variação do angulo $\theta$ em cada iteração para se obter os pontos.

Tal como o plano, um círculo também tem uma orientação. Caso se pretenda desenhar o círculo virado no sentido positivo do eixo dos Y, a ordem de colocação dos pontos deverá ser C-A-B. Caso se pretenda desenhar o círculo virado no sentido negativo do eixo dos Y, a ordem de colocação dos pontos deverá ser C-B-A.

A função responsável pelo desenho de círculos é a função \textit{criaCirculo()}, cujo pseudo-código é o seguinte:

\begin{Verbatim}
Figura& criaCirculo(Ponto3D C, float raio, int fatias, 
			int orientacao) {

	CoordsPolares A, B;
	float dAz = (2.0 * M_PI) / (fatias + 0.0f);
	
	if (orientacao == 1) {
		for (int i = 0; i < fatias; i++) {
			A = CoordsPolares(C, raio, dAz*i);
			B = CoordsPolares(C, raio, dAz*(i + 1));
			
			Gera pontos pela ordem C-A-B
		}
	}
	else {
		for (int i = 0; i < fatias; i++) {
			A = CoordsPolares(C, raio, dAz*i);
			B = CoordsPolares(C, raio, dAz*(i + 1));
			
			Gera pontos pela ordem C-A-B
		}
	}
	
	
	return *this;
}



\end{Verbatim}

\newpage
\subsection{Gerar Cone}

Para gerar um cone, o utilizador deve efetuar o comando com a seguinte sintaxe:

\begin{Verbatim}
Gerador cone raio altura fatias camadas ficheiro
\end{Verbatim}

A base do cone corresponde a um círculo virado no sentido negativo dom eixo dos Y. Como apresentado na secção \ref{p3:circulo}, a função \textit{criaCirculo()} permite fazer isto. Resta apenas gerar os pontos para a superfície ``curva'' do cone. Para essa superfício, abordagem seguida foi a de fazer o cone camada a camada. As fatias partem cada camada numa espécie ``rectângulos'' como o que se mostra na figura \ref{p1:fig:p3_seccaoCone_edit}.


\begin{figure}[<+htpb+>]
	\centering
	\includegraphics[scale=0.5]{imagens/p3_seccaoCone_edit.png}
	\caption{Esquema representativo de uma fatia de uma camada do cone (Pontos A,B,C e D)}
	\label{p1:fig:p3_seccaoCone_edit}
\end{figure}

Ou seja, é possível construir uma fatia de uma dada camada com os pontos A,B,C e D. É no entanto necessário saber as coordenadas destes pontos. Para tal, foram usadas coordenadas polares, com auxílio mais uma vez da classe \textit{CoordsPolares}. Podemos ver os pontos A e D como pertencentes a um mesmo círculo, com centro no centro do cone. Da mesma forma, também os pontos B e C se encontram num mesmo círculo, com raio maior do que o círculo onde estão os pontos A e D. O raio dos dois círculos em que estes dois conjuntos de pontos se encontram difere, no entanto é fácil perceber que o raio do círculo depende da camada que se está a considerar. Por exemplo, se virmos a base do cone como um círculo de raio $r$, então o círculo correspondente à camada de cima terá raio $r - (r/camadas)$. A figura \ref{p1:fig:p3_conePerfil} pretende ilustrar tal situação para um exemplo de 3 camadas.

\begin{figure}[<+htpb+>]
	\centering
	\includegraphics[scale=0.5]{imagens/p3_conePerfil.png}
	\caption{O círculo da camada superior à que se considera tem sempre raio $r - (r/camadas)$}
	\label{p1:fig:p3_conePerfil}
\end{figure}

Saber o centro de cada um dos círculos do cone também é simples. Os centros dos círculos encontram-se todos no centro do cone, a única coisa que varia em cada um é a coordenada Y cuja diferença para a coordenada Y da camada anterior é $altura / camadas$.

Visto que a classe de \textit{CoordsPolares} é capaz de dar coordenadas cartesianas sendo indicado um centro, um raio e um ângulo e tendo em conta o exposto anteriormente então a função criaCone() fica simplesmente:

\begin{Verbatim}
Figura& criaCone(Ponto3D centroCone, float altura, 
		float raio, int camadas, int fatias) {
	
	float dRaio = raio / (camadas + 0.0f);
	float dAltura = altura / (camadas + 0.0f);
	float dAz = (2.0 * M_PI) / (fatias + 0.0f);
	float meiaAltura = altura / 2.0f;
	
	Calcula coordendadas do centro da base
	
	Cria circulo no centro da base virado no sentido
	negativo do eixo dos Y.
	
	for (int i = 0; i < camadas; i++) {
		for (int j = 0; j < fatias; j++) {
			Calcula o centro do circulo "de baixo" da
			camada que se está a considerar
			
			Calcula o centro do circulo "de cima" da
			camada que se está a considerar
			
			A = CoordsPolares(cCima,
					raio - (dRaio*(i + 1.0)),
					dAz * j);
			B = CoordsPolares(cBaixo,
					raio - (dRaio*(i + 0.0)),
					dAz * j);
			C = CoordsPolares(cBaixo,
					raio - (dRaio*(i + 0.0)),
					dAz * (j + 1.0));
			D = CoordsPolares(cCima,
					raio - (dRaio*(i + 1.0)),
					dAz * (j + 1.0));
			
			Gera os pontos pela ordem A-B-C-C-D-A
		}
	}
	
	return *this;
}
\end{Verbatim}

\begin{figure}[<+htpb+>]
	\centering
	\includegraphics[scale=0.5]{imagens/p3_cone_2_3_10_10.png}
	\caption{Exemplo de cone gerado, com 2 de raio de base, 3 de altura, 10 camadas e 10 fatias}
	\label{p1:fig:p3_cone_2_3_10_10}
\end{figure}

\newpage
\subsection{Gerar Esfera}

Para gerar uma esfera, o utilizador deve efetuar o comando com a seguinte sintaxe:

\begin{Verbatim}
Gerador sphere raio fatias camadas ficheiro
\end{Verbatim}

Os pontos da esfera foram gerados à custa de coordenadas esféricas, nomeadamente com o auxílio da classe \textit{CoordsEsfericas}.

À semehança do cone também para a esfera é possível gerar uma fatia de uma determinada camada usando 4 pontos conforme mostrado na figura \ref{p1:fig:p3_esferaSeccao_edit}.

\begin{figure}[<+htpb+>]
	\centering
	\includegraphics[scale=0.5]{imagens/p3_esferaSeccao_edit.png}
	\caption{Esquema representativo de uma fatia de uma camada da esfera (Pontos A,B,C e D)}
	\label{p1:fig:p3_esferaSeccao_edit}
\end{figure}

Também para a esfera a abordagem seguida foi a de construir camada a camada, começando pela camada de cima. A determinação das coordenadas dos pontos A, B, C e D é simples. Uma vez que se está a usar a classe \textit{CoordsEsfericas}, a única coisa que se tem que perceber para usar a classe é de que forma variam os ângulos polar e azimuth entre camadas e fatias. Ora, tendo $c$ camadas, sabe-se que a diferença em termos de ângulo polar de um ponto que esteja numa camada superior para outro que esteja numa camada imediatamente inferior é de $\pi / c$. Por outro lado, se considerarmos $f$ fatias, a diferença do ângulo azimuth de um ponto para outro que esteja numa fatia imediatamente a seguir é de $(2 \times \pi) / f$. Assim temos o seguinte pseudo-código:

\begin{Verbatim}
Figura& criaEsfera(int fatias, int camadas, float raio) {
	
	float deltaPolar = M_PI / (camadas+0.0f);
	float deltaAz = (2.0 * M_PI) / (fatias+0.0f);
	
	for (int i = 0; i < camadas; i++) {
		for (int j = 0; j < fatias; j++) {
			A = CoordsEsfericas(raio, 
					deltaAz * (j + 0.0f), 
					deltaPolar * (i+0.0f));
			B = CoordsEsfericas(raio,
					deltaAz * (j + 0.0f), 
					deltaPolar * (i+1.0f));
			C = CoordsEsfericas(raio,
					deltaAz * (j + 1.0f),
					deltaPolar * (i+1.0f));
			D = CoordsEsfericas(raio,
					deltaAz * (j + 1.0f),
					deltaPolar * (i+0.0f));
			
			Gera pontos pela ordem A-B-C-C-D-A
			
		}
	}
	
	return *this;
}
\end{Verbatim}

\begin{figure}[<+htpb+>]
	\centering
	\includegraphics[scale=0.5]{imagens/p3_esfera_2_20_20.png}
	\caption{Exemplo de esfera gerada, com 2 de raio, 20 camadas e 20 fatias}
	\label{p1:fig:p3_esfera_2_20_20}
\end{figure}

