\chapter{Introdução}
\label{cap:intro}

No âmbito da Unidade Curricular de Computação Gráfica pertencente ao plano de estudos do 3º ano do Mestrado Integrado em Engenharia Informática, foi proposto o desenvolvimento de um mini motor 3D. 


 Nesta primeira fase o motor 3D será o responsável pelo desenho dos modelos 3D armazenados em ficheiros diferentes. Para além do motor será criado um gerador que recebe como parâmetros o nome das primitivas gráficas e os dados necessários à sua criação. O gerador deverá escrever num ficheiro os pontos indispensáveis ao desenho da primitiva.
 
 O gerador deverá ser capaz de gerar um plano, um cubo (caixa), um cone e uma esfera. Os ficheiros gerados deverão conter os pontos dos triângulos, em que cada linha corresponde a um ponto e a cada 3 linhas a um triângulo.
