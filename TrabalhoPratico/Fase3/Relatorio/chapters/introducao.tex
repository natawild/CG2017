\chapter{Introdução}
\label{cap:intro}

No âmbito da Unidade Curricular de Computação Gráfica pertencente ao plano de estudos do 3º ano do Mestrado Integrado em Engenharia Informática foi proposto o desenvolvimento de um sistema solar. 

Concluída a segunda fase tem-se agora de introduzir VBOS, este método consegue aumentar consideravelmente a performance de uma aplicação pois reduz o número de acessos à memória do computador, visto que todos os pontos são carregados para um buffer no CPU, sendo isto dos pontos mais importantes desta terceira fase do projeto. 

Por forma a tornar o sistema solar mais próximo do real foram implementadas funcionalidades no \textit{engine} que permitem definir pontos de controlo de uma curva de \textit{Catmull-Rom} para um dado grupo, sendo assim possível criar curvas que os planetas devem percorrer de forma a realizar a translação à volta do sol. 

Além dos pontos de controlo da curva que devem seguir também é  possivel definir quanto tempo a translação deverá demorar. O último ponto desenvolvido é a utilização de superfícies de \textit{Bezier}.
