\chapter{Conclusão}

Para a realizaçãoo desta primeira fase tivemos como objectivo desenvolver um gerador de figuras com capacidade de gerar pontos para um plano, caixa, cone e esfera.  Além destas figuras foram ainda desenvolvidas funções adicionais que permitem criar planos sem ser no eixo XZ. 

No lado do \textit{engine}, a aplicação consegue ler ficheiros XML e .3D e a partir dai desenhar as figuras com os pontos especificados nos ficheiros. Adicionalmente ao sugerido, incluiu-se também uma câmara colocada sobre uma esfera que permite ao utilizador ver a figura desenhada de vários ângulos. 


Não obstante, existem aspectos em que o trabalho que poderiam ser melhorados e serão alvo de atenção no futuro. Pretendemos que a interação com a camara seja melhorada, pois currentemente a sua interação é pouco intuitiva. 


A prioridade desde cedo foi ter código simples, funcional e fácil de ler. Tal implicou que muitas vezes quando confrontados com a decisão de manter algumas variáveis como públicas ou privadas a decisão tenha sido manter públicas. 



Existem, no entanto, alguns problemas de otimizaçãoo e desorganização
no código. Ou seja, existem algumas linhas de código que como estão a ser
repetidas no programa deveriam passar para funções distintas. 
Como o objetivo principal do projeto não é ter bons módulos de dados nem bom encapsulamento dos mesmos, estes aspectos foram deixados para segundo plano, no entanto serão alvo de uma atenção mais cuidada no futuro.

Em suma, achamos que, com esta fase 1, foram criadas as bases para a
realização das demais fases.
