Para tal criou-se uma nova classe Material para albergar todas as informações necessárias que são as cores RGB para as componentes ambiente, difusa, especular e emissiva.
O XMLParser proporciona portanto novas funções para extrair cada uma das componentes do .xml, colocando como placeholders 0.2 para cada cor ambiente, 0.8 para difusa, e 0 para as restantes. Para o sol utilizou-se as propriedades difusas a 0 e as emissivas a 1 para
poder ser visível apesar da luz estar no seu interior e portanto não o afetar.

Apesar destes pequenos pontos em falta real¸cam-se alguns pontos, como por exemplo, o facto de ser possível utilizar uma imagem como background da cena, imagem essa que é aplicada a uma esfera que engloba a câmara e se move com a mesma. 

Para além disso a performance do engine é boa, já que o número de fps ronda o limite de 60 quando está correr o sistema solar.


\subsection{Superficies de Bezier}

O vector da normal em qualquer ponto da Superfície é definido pelo produto cruzado dos vetores das tangentes normalizadas.”, ou seja, começa-se por calcular as tangentes em u e em v normalizando-se em seguida os vetores, por fim, faz-se o produto cruzado entre os dois vetores e obtem-se uma normal.


Para calcular as tangentes usa-se uma função que recebe como argumentos o valor de u e v a matriz m os pontos da patch de bezier e um inteiro de serve para indicar se estamos a calcular em u ou em v, esta funçao devolve-nos um ponto, referente a x, y ou z que é passado na matrix p.
Cross e normalize, assim como o nome indica são as funções do cálculo do produto cruzado e da normalização. De realçar que para a normalização, verifica-se se o divisor é diferente de 0, pois se for 0 será devolvido o erro de not a number, pois é impossivel dividir por zero.
A funçãoo bezierTangent calcula os diversos pontos em u e em v. De realçar que os pontos são guardados numa queue de nome derivada para serem posteriormente escritos no ficheiro .3d.


As texturas são só obtidas quando no xml se pode encontrar um atributo texture num \textit{file} e para as representar deveria-se ter utilizado uma classe Texture para não sobrecarregar o Group.